\documentclass[notes,10pt,aspectratio=169]{beamer}

\usepackage{pgfpages}
% These slides also contain speaker notes. You can print just the slides,
% just the notes, or both, depending on the setting below. Comment out the want
% you want.
\setbeameroption{hide notes} % Only slide
%\setbeameroption{show only notes} % Only notes
%\setbeameroption{show notes on second screen=right} % Both

\usepackage{array}
\usepackage{tgbonum}

% The magic happens here
\usepackage{fontspec}
% \usepackage{lato}
% The magic happens here
\setsansfont[ItalicFont={Fira Sans Italic},%
BoldFont={Fira Sans Bold},%
BoldItalicFont={Fira Sans Italic}]%
{Fira Sans Light}%

\usepackage{tikz}
\usepackage{verbatim}
\setbeamertemplate{note page}{\pagecolor{yellow!5}\insertnote}
\usetikzlibrary{positioning}
\usetikzlibrary{snakes}
\usetikzlibrary{calc}
\usetikzlibrary{arrows}
\usetikzlibrary{decorations.markings}
\usetikzlibrary{shapes.misc}
\usetikzlibrary{matrix,shapes,arrows,fit,tikzmark}
\usepackage{amsmath}
\usepackage{mathpazo}
\usepackage{hyperref}
\usepackage{lipsum}
\usepackage{multimedia}
\usepackage{graphicx}
\usepackage{multirow}
\usepackage{graphicx}
\usepackage{dcolumn}
\usepackage{bbm}
\newcolumntype{d}[0]{D{.}{.}{5}}

\usepackage{changepage}
\usepackage{appendixnumberbeamer}
\newcommand{\beginbackup}{
   \newcounter{framenumbervorappendix}
   \setcounter{framenumbervorappendix}{\value{framenumber}}
   \setbeamertemplate{footline}
   {
     \leavevmode%
     \hline
     box{%
       \begin{beamercolorbox}[wd=\paperwidth,ht=2.25ex,dp=1ex,right]{footlinecolor}%
%         \insertframenumber  \hspace*{2ex} 
       \end{beamercolorbox}}%
     \vskip0pt%
   }
 }
\newcommand{\backupend}{
   \addtocounter{framenumbervorappendix}{-\value{framenumber}}
   \addtocounter{framenumber}{\value{framenumbervorappendix}} 
}

\usepackage{graphicx}
\usepackage[space]{grffile}
\usepackage{booktabs}

\usepackage{verbatim}
\usepackage{ragged2e}
\usepackage{etoolbox}
\apptocmd{\frame}{}{\justifying}{} % Allow optional arguments after frame.

% These are my colors -- there are many like them, but these ones are mine.
\definecolor{blue}{RGB}{0,114,178}
\definecolor{red}{RGB}{213,94,0}
\definecolor{yellow}{RGB}{240,228,66}
\definecolor{green}{RGB}{0,158,115}
\definecolor{crimson}{RGB}{165,28,48}

\hypersetup{
  colorlinks=crimson,
  urlcolor=crimson,
  linkbordercolor = {white},
  linkcolor = {blue}
}

%% I use a beige off white for my background
\definecolor{MyBackground}{RGB}{255,253,218}

%% Uncomment this if you want to change the background color to something else
%\setbeamercolor{background canvas}{bg=MyBackground}

%% Change the bg color to adjust your transition slide background color!
\newenvironment{transitionframe}{
  \setbeamercolor{background canvas}{bg=yellow}
  \begin{frame}}{
    \end{frame}
}

\setbeamercolor{frametitle}{fg=crimson}
\setbeamerfont{frametitle}{series=\bfseries}
\setbeamercolor{title}{fg=black}
\setbeamerfont{title}{series=\bfseries}
\setbeamertemplate{footline}[frame number]
\setbeamertemplate{navigation symbols}{} 
\setbeamertemplate{itemize items}{-}
\setbeamercolor{itemize item}{fg=crimson}
\setbeamercolor{itemize subitem}{fg=crimson}
\setbeamercolor{enumerate item}{fg=crimson}
\setbeamercolor{enumerate subitem}{fg=crimson}
\setbeamercolor{button}{bg=MyBackground,fg=crimson,}


% If you like road maps, rather than having clutter at the top, have a roadmap show up at the end of each section 
% (and after your introduction)
% Uncomment this is if you want the roadmap!
% \AtBeginSection[]
% {
%    \begin{frame}
%        \frametitle{Roadmap of Talk}
%        \tableofcontents[currentsection]
%    \end{frame}
% }
\setbeamercolor{section in toc}{fg=blue}
\setbeamercolor{subsection in toc}{fg=red}
\setbeamersize{text margin left=1em,text margin right=1em} 

\newenvironment{wideitemize}{\itemize\addtolength{\itemsep}{10pt}}{\enditemize}

\usepackage{environ}
\NewEnviron{videoframe}[1]{
  \begin{frame}
    \vspace{-8pt}
    \begin{columns}[onlytextwidth, T] % align columns
      \begin{column}{.70\textwidth}
        \begin{minipage}[t][\textheight][t]
          {\dimexpr\textwidth}
          \vspace{8pt}
          \hspace{4pt} {\Large \sc \textcolor{blue}{#1}}
          \vspace{8pt}
          
          \BODY
        \end{minipage}
      \end{column}%
      \hfill%
      \begin{column}{.38\textwidth}
        \colorbox{green!20}{\begin{minipage}[t][1.2\textheight][t]
            {\dimexpr\textwidth}
            Face goes here
          \end{minipage}}
      \end{column}%
    \end{columns}
  \end{frame}
}

\renewcommand{\appendixname}{\texorpdfstring{\translate{Appendix}}{Appendix}}
% Remove footnote rule
\renewcommand\footnoterule{}

\AtBeginSection{
  \begingroup
  \setbeamercolor{background canvas}{bg=crimson}
  \setbeamercolor{title}{fg=white}
  \setbeamertemplate{footline}{} % Remove page numbering
  \frame{\centering\Large\bf\textcolor{white}{\insertsectionhead}}
  \endgroup
}

\title[]{\textcolor{crimson}{Sesión 3: RDD}}
\author[RRMRR]{Rony Rodrigo Maximiliano Rodriguez-Ramirez}
\institute[Harvard]{\small{Econ Thaki \\ Harvard University}}
\date{\today}


\begin{document}

%%% TIKZ STUFF
\tikzset{   
        every picture/.style={remember picture,baseline},
        every node/.style={anchor=base,align=center,outer sep=1.5pt},
        every path/.style={thick},
        }
\newcommand\marktopleft[1]{%
    \tikz[overlay,remember picture] 
        \node (marker-#1-a) at (-.3em,.3em) {};%
}
\newcommand\markbottomright[2]{%
    \tikz[overlay,remember picture] 
        \node (marker-#1-b) at (0em,0em) {};%
}
\tikzstyle{every picture}+=[remember picture] 
\tikzstyle{mybox} =[draw=black, very thick, rectangle, inner sep=10pt, inner ysep=20pt]
\tikzstyle{fancytitle} =[draw=black,fill=red, text=white]
%%%% END TIKZ STUFF

% logo of my university
\titlegraphic{
  \includegraphics[width=2cm]{logo/logo_hq.png}\hspace*{4.75cm}~%
  \includegraphics[width=2cm]{logo/harvard-logo.png}
}
% \titlegraphic{
%   \includegraphics[width=2.5cm]{logo/harvard-logo.png}
% }


% Title Slide --------------------------------------------------------------------------------------
\begin{frame}[noframenumbering, plain]
  \maketitle
\end{frame}
% --------------------------------------------------------------------------------------------------

% Diapositiva de contenido -------------------------------------------------------------------------
\begin{frame}
  \frametitle{Agenda para hoy}

  En la sesión de hoy, cubriremos:
  \begin{enumerate}
    \item Introducción al Diseño de Discontinuidad en la Regresión (RDD)
    \item RDD Estricto y Configuración Básica
    \item Supuestos e Interpretación
    \item RDD Difuso
    \item Ejemplos y Demostración en Stata
  \end{enumerate}
\end{frame}

% Diapositiva 1: Introducción al RDD ---------------------------------------------------------------
\begin{frame}
  \frametitle{Introducción al RDD}

  \begin{itemize}
    \item \textbf{El Diseño de Discontinuidad en la Regresión (RDD)} es un diseño cuasi-experimental que busca encontrar el efecto causal de las intervenciones asignando un punto de corte por encima o por debajo del cual se asigna una intervención.

    \vspace{1em}

    \item \textbf{Importancia:}
    \begin{itemize}
      \item Útil en la evaluación de políticas cuando la asignación aleatoria no es posible.
      \item Proporciona inferencia causal robusta en estudios observacionales.
    \end{itemize}
  \end{itemize}

\end{frame}

% Diapositiva 2: Introducción al RDD ---------------------------------------------------------------
\begin{frame}
  \frametitle{Introducción al RDD}

	\begin{itemize}
		\item Presentado por primera vez para estudiar el impacto del reconocimiento al mérito por Thistlethwaite \& Campbell (1960).
		\item Sin embargo, solo comenzó a llamar la atención en economía desde finales de la década de 1990.
		\item Pero ¿qué es una \textbf{discontinuidad}?
		\begin{itemize}
			\item Una ruptura brusca en los valores de una función.
			\item Matemáticamente, estamos hablando de una ecuación por partes (piecewise equation):
			$$
			f(x) = \left\{
			\begin{array}{ll}
			\frac{1}{2} x, 	& \quad x < 5 \\
			2+\frac{1}{2}x,  & \quad x \geq 5
			\end{array}
			\right.
			$$
		\end{itemize}
	\end{itemize}
\end{frame}

% Diapositiva 2: RDD Estricto y Configuración Básica -----------------------------------------------
\begin{frame}
  \frametitle{RDD Estricto y Configuración Básica}

  \textbf{RDD Estricto:} 
  
  La asignación del tratamiento se determina perfectamente por si la covariable observada supera o no un umbral conocido.

  \vspace{1em}
  \textbf{Configuración Básica:}
  \begin{itemize}
    \item Variable de Corte: La variable continua que determina la asignación del tratamiento.
    \item Punto de Corte: El valor de la variable de corte donde cambia la asignación del tratamiento.
  \end{itemize}
  \begin{equation}
    Y_i = \alpha + \tau D_i + \beta X_i + \epsilon_i
  \end{equation}
  donde $Y_i$ es el resultado, $D_i$ es el indicador de tratamiento, $X_i$ es la variable de corte, y $\epsilon_i$ es el término de error.
\end{frame}

% Diapositiva 3: Supuestos e Interpretación 
\begin{frame}
  \frametitle{Supuestos e Interpretación}

  \textbf{Supuestos:}
  \begin{itemize}
    \item Continuidad: Los resultados potenciales son continuos en el punto de corte.
    \item Sin manipulación: La variable de corte no puede ser manipulada de manera precisa.
  \end{itemize}

  \vspace{1em}
  \textbf{Interpretación:}
  \begin{itemize}
    \item El efecto del tratamiento $\tau$ se interpreta como el efecto local promedio del tratamiento en el punto de corte.
    \item Visualizar la discontinuidad en la variable de resultado en el punto de corte para inferir el efecto causal.
  \end{itemize}
\end{frame}

% Diapositiva 4: RDD Difuso
\begin{frame}
  \frametitle{RDD Difuso}

  \textbf{RDD Difuso:} 
  
  La asignación del tratamiento no está determinada perfectamente por el punto de corte; en cambio, hay una probabilidad de asignación que cambia en el punto de corte.

  \vspace{1em}
  \begin{itemize}
    \item Requiere un enfoque de variable instrumental para estimar el efecto del tratamiento.
    \item Se utiliza comúnmente el método de mínimos cuadrados en dos etapas (2SLS).
  \end{itemize}
  \begin{equation}
    Y_i = \alpha + \tau Z_i + \beta X_i + \epsilon_i
  \end{equation}
  donde $Z_i$ es la variable instrumental que indica la probabilidad del tratamiento.
\end{frame}

% Diapositiva 5: Lista de Verificación y Ejemplos
\begin{frame}
  \frametitle{Lista de Verificación y Ejemplos}

  \textbf{Lista de Verificación para RDD:}
  \begin{itemize}
    \item Verificar la continuidad de las covariables en el punto de corte.
    \item Comprobar la manipulación de la variable de corte.
    \item Utilizar el análisis gráfico para visualizar la discontinuidad.
  \end{itemize}

  \vspace{1em}
  \textbf{Ejemplos:}
  \begin{itemize}
    \item Evaluación del impacto de un programa de becas basado en puntajes de exámenes.
    \item Evaluación del efecto de un cambio de política en la asistencia escolar.
  \end{itemize}
\end{frame}

% Diapositiva 6: Demostración en Stata
\begin{frame}[fragile]
  \frametitle{Demostración en Stata}

  En Stata, el RDD se puede implementar usando comandos como \texttt{rdrobust} o \texttt{rddensity}.

  \vspace{1em}
  \textbf{Código de Ejemplo:}
  \begin{verbatim}
    rdrobust Y X, c(0)
    rddensity X, c(0)
  \end{verbatim}

  \vspace{1em}
  \textbf{Interpretación de los Resultados:}
  \begin{itemize}
    \item Las estimaciones de los coeficientes proporcionan el efecto del tratamiento en el punto de corte.
    \item Utilizar los gráficos de salida para apoyar la inferencia causal.
  \end{itemize}
\end{frame}

% End ----------------------------------------------------------------------------------------------
\end{document}