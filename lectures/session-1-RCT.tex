\documentclass[11pt,a4paper,english]{article}  
\usepackage[utf8]{inputenc}
\usepackage[sc]{mathpazo}
\usepackage{amsmath}
\usepackage[top=2.5cm, bottom=2.5cm, left=2.25cm, right=2.25cm,footskip=1.5cm]{geometry}  

\title{Sesión 1 - RCT}
\author{}
\date{}

\begin{document}

\maketitle

\section*{Para esta sesión, seguiremos el artículo:} Fafchamps, Marcel, David
McKenzie, Simon Quinn y Christopher Woodruff. 2014. “Microenterprise growth and
the flypaper effect: Evidence from a randomized experiment in Ghana.” Journal of
Development Economics, 106: 211-226. 

\section*{Preguntas Conceptuales}

\begin{enumerate}
    \item ¿Cuál es la pregunta principal de investigación que los autores abordan en este artículo? Explica por qué esta pregunta es significativa.
    \item Resume el principal hallazgo del artículo, incluyendo una explicación del "efecto flypaper," utilizando un lenguaje no técnico.
    \item Describe el tratamiento específico que recibieron las empresas en el país del estudio.
    \item Define la diferencia entre la Intención de Tratar (ITT) y el Tratamiento sobre los Tratados (TOT) en el contexto de este estudio, donde el tratamiento se define como la asignación al programa de subvenciones. ¿Qué medida reportan los autores y por qué? Proporciona una ecuación utilizando la notación de resultados potenciales para ilustrar lo que los autores intentan estimar.
    \item ¿A qué nivel agrupan los errores estándar los autores para los principales resultados del artículo (si corresponde)? Explica brevemente por qué se utiliza la agrupación y por qué este nivel es apropiado.
    \item Para verificar si el tratamiento se asignó aleatoriamente, examinamos los resultados de la prueba de balance en la Tabla 2. ¿Estos resultados aumentan o disminuyen tu confianza en los hallazgos del artículo? Interpreta uno de los valores p de la columna (5) para la muestra completa.
\end{enumerate}

\section*{Preguntas de Análisis de Datos}

En esta sección, replicarás los resultados centrales del artículo analizando los datos proporcionados y presentando tus hallazgos. El enlace del conjunto de problemas incluye una versión ligeramente limpiada de los archivos principales de análisis: ReplicationDataGhanaJDE.dta.

\begin{enumerate}
    \item Produce una tabla de estadísticas descriptivas bien organizada para la Ola 2 que incluya: (i) el número de hogares, (ii) el número de unidades de randomización, (iii) la media muestral y el error estándar del beneficio final real en el grupo de control, y (iv) la misma media muestral y error estándar en el grupo de tratamiento. La tabla debe tener una fila por país y cinco columnas, incluyendo el nombre del país.
    \item Verifica la randomización como se hizo en la Tabla 2 del artículo. Interpreta el valor p de las estadísticas F en la primera fila. ¿Qué está indicando?
    \item Reproduce las estimaciones de coeficientes y errores estándar de las columnas 1-6 de la Tabla 3 del artículo y exporta los resultados a una tabla bien formateada en LaTeX.
    \item Reproduce las estimaciones de coeficientes y errores estándar de las columnas 1-6 de la Tabla 3 del artículo, agrega las estadísticas de prueba al final de la tabla, y exporta los resultados a una tabla bien formateada en LaTeX.
\end{enumerate}

\end{document}
